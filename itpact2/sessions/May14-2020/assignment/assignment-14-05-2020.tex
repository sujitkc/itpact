\documentclass[addpoints,11pt]{exam}
\usepackage{mathpartir}
\usepackage{framed}
\usepackage{graphicx}
\usepackage[colorlinks]{hyperref}
\usepackage{listings}
\usepackage{xcolor}
\usepackage{tikz}
\usepackage{crayola}
\usetikzlibrary{positioning,shapes,arrows,shadows}
\usetikzlibrary{automata,positioning,shapes,arrows,backgrounds,fit}

\author{}
\title{ITPACT \\ Assignment May 14, 2020}

\date{}
\definecolor{lightblue}{rgb}{0.8,0.93,1.0} % color values Red, Green, Blue
\definecolor{Blue}{rgb}{0,0,1.0} % color values Red, Green, Blue
\definecolor{Red}{rgb}{1,0,0} % color values Red, Green, Blue
\definecolor{Pink}{rgb}{0.7,0,0.2}
\definecolor{links}{HTML}{2A1B81}
\definecolor{mydarkgreen}{HTML}{126215}
\definecolor{mymaroon}{HTML}{800000}

\newcommand{\myprod}[0]{\hspace{0.5cm}$::=$\hspace{0.5cm}}
\newcommand{\mychoice}[0]{\hspace{0.75cm}$|$\hspace{0.25cm}}
\newcommand{\grade}[1]{{\color{red}\textbf{Grading Key:} #1}}

\let\oldhref\href
\renewcommand{\href}[2]{\oldhref{#1}{\bfseries#2}}

\hypersetup{
    colorlinks,
    citecolor=black,
    filecolor=black,
    linkcolor=mymaroon,
    urlcolor=mydarkgreen
}
\lstdefinestyle{pc}{
	language = Python,
	basicstyle = \scriptsize\ttfamily,
	stringstyle = \ttfamily\color{Purple},
	keywordstyle=\color{black}\bfseries,
	identifierstyle=\ttfamily\color{BrickRed},
	frame=single,
	frameround=tttt,
	frame=single,
	numbers=none,
	showstringspaces=false
}

\lstdefinestyle{jc}{
	language = Java,
	basicstyle = \small\ttfamily,
	stringstyle = \ttfamily,
	keywordstyle=\color{Blue}\bfseries,
	identifierstyle=\color{Pink},
	commentstyle=\color{OliveGreen},
	frame=single,
	frameround=tttt,
	showstringspaces=false
}

\lstdefinestyle{occ}{
	language = [Objective]Caml,
	basicstyle = \small\ttfamily,
	stringstyle = \color{red}\ttfamily,
	keywordstyle=\color{Blue}\bfseries,
	identifierstyle=\color{BrickRed}\ttfamily,
	frameround=tttt,
	frame=single,
	numbers=none,
	showstringspaces=false,
	mathescape=true,
	escapeinside={(*@}{@*)}
}

\lstdefinestyle{oc}{
	language = bash,
	backgroundcolor = \color{black!60},
	basicstyle = \small\ttfamily\color{white},
	stringstyle = \color{red}\ttfamily,
	keywordstyle=\color{white}\bfseries,
	identifierstyle=\ttfamily,
	commentstyle=\color{Yellow},
	frameround=tttt,
	frame=single,
	numbers=none,
	showstringspaces=false,
	escapeinside={(*@}{@*)}
}

\begin{document}
  \tikzstyle{fun} = [draw=gray, thick, fill=white, rounded corners, rectangle, inner sep=1cm]
\maketitle

\pointsinrightmargin

\thispagestyle{head}

\section*{Questions:}

\begin{questions}

\question
Once upon a time there were three friends -- Amar, Akbar and Anthony.
They decided to go for a movie outing. The expenses were shared as follows:

\begin{center}
\begin{tabular}{ccc}
\hline
  ITEM         & EXPENSE   &       CONTRIBUTOR \\
\hline
Transportation &     1000  &           Amar    \\
Movie tickets  &     1200  &           Akbar   \\
Dinner         &     2500  &           Anthony \\
Ice cream      &      500  &           Akbar   \\
Popcorn        &      400  &           Amar  \\
\hline
\end{tabular}
\end{center}

Use Python interpreter to do the following calculations:
\begin{enumerate}
\item What was the total money spent?
\item How much did each friend spend on average?
\item How much does each friend owe/receive at the end of day?
\end{enumerate}

Submit the dump of your interpreter interaction.


\textbf{\color{BrickRed}Note:} \\
For example, the dump of the interpreter interaction that we did today in the
session is as follows:
\begin{lstlisting}[style=oc]
*******************************************************************************
>>> one = 1 * 20
>>> two = 2 * 17
>>> five = 5 * 15
>>> ten = 10 * 15
>>> twenty = 20 * 6
>>> fifty = 50 * 43
>>> hundred = 100 * 21
>>> five_hundred = 500 * 36
>>> one + two + five + ten + twenty + fifty + hundred + five_hundred
22649
*******************************************************************************
\end{lstlisting}

\end{questions}
\end{document}