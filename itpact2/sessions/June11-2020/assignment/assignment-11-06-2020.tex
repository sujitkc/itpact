\documentclass[addpoints,11pt]{exam}
\usepackage{mathpartir}
\usepackage{framed}
\usepackage{graphicx}
\usepackage[colorlinks]{hyperref}
\usepackage{listings}
\usepackage{xcolor}
\usepackage{tikz}
\usepackage{crayola}
\usetikzlibrary{positioning,shapes,arrows,shadows}
\usetikzlibrary{automata,positioning,shapes,arrows,backgrounds,fit}

\author{}
\title{ITPACT \\ Assignment June 11, 2020}

\date{}
\definecolor{lightblue}{rgb}{0.8,0.93,1.0} % color values Red, Green, Blue
\definecolor{Blue}{rgb}{0,0,1.0} % color values Red, Green, Blue
\definecolor{Red}{rgb}{1,0,0} % color values Red, Green, Blue
\definecolor{Pink}{rgb}{0.7,0,0.2}
\definecolor{links}{HTML}{2A1B81}
\definecolor{mydarkgreen}{HTML}{126215}
\definecolor{mymaroon}{HTML}{800000}

\newcommand{\kctt}[1]{\lstinline[basicstyle=\ttfamily]@#1@}
\newcommand{\kcsecdiv}[0]{
\begin{center}
\begin{tikzpicture}
\draw[-] (0,0) -- (1.8, 0);
\draw (2, 0) circle (0.1cm);
\draw[-] (2.2,0) -- (4, 0);
\end{tikzpicture}
\end{center}
}
\let\oldhref\href
\renewcommand{\href}[2]{\oldhref{#1}{\bfseries#2}}

\hypersetup{
    colorlinks,
    citecolor=black,
    filecolor=black,
    linkcolor=mymaroon,
    urlcolor=mydarkgreen
}
\lstdefinestyle{pc}{
	language = Python,
	basicstyle = \scriptsize\ttfamily,
	stringstyle = \ttfamily\color{Purple},
	keywordstyle=\color{black}\bfseries,
	identifierstyle=\ttfamily\color{BrickRed},
	frame=single,
	frameround=tttt,
	frame=single,
	numbers=none,
	showstringspaces=false
}

\lstdefinestyle{jc}{
	language = Java,
	basicstyle = \small\ttfamily,
	stringstyle = \ttfamily,
	keywordstyle=\color{Blue}\bfseries,
	identifierstyle=\color{Pink},
	commentstyle=\color{OliveGreen},
	frame=single,
	frameround=tttt,
	showstringspaces=false
}

\lstdefinestyle{occ}{
	language = [Objective]Caml,
	basicstyle = \small\ttfamily,
	stringstyle = \color{red}\ttfamily,
	keywordstyle=\color{Blue}\bfseries,
	identifierstyle=\color{BrickRed}\ttfamily,
	frameround=tttt,
	frame=single,
	numbers=none,
	showstringspaces=false,
	mathescape=true,
	escapeinside={(*@}{@*)}
}

\lstdefinestyle{oc}{
	language = bash,
	backgroundcolor = \color{black!60},
	basicstyle = \small\ttfamily\color{white},
	stringstyle = \color{red}\ttfamily,
	keywordstyle=\color{white}\bfseries,
	identifierstyle=\ttfamily,
	commentstyle=\color{Yellow},
	frameround=tttt,
	frame=single,
	numbers=none,
	showstringspaces=false,
	escapeinside={(*@}{@*)}
}

\begin{document}
  \tikzstyle{fun} = [draw=gray, thick, fill=white, rounded corners, rectangle, inner sep=1cm]
\maketitle

\pointsinrightmargin

\thispagestyle{head}

\section*{Instructions}
\begin{enumerate}
\item All problems should be solved using loops.
\item In all the problems given below, please try to create the solution by using
    only the features of the programming language which have been discussed
    in the class so far.
\item For all the problems below, use for loops.
\end{enumerate}

\kcsecdiv

\begin{questions}
\question Write a program that prints all elements of a list in the same order. User for loop.

  (\textbf{\color{BrickRed}File:} \kctt{print_list.py})

\question Write a program that prints all elements of a list in reverse order. User for loop.

  (\textbf{\color{BrickRed}File:} \kctt{print_rev_list.py})

\question Write a program that computes and prints the product of all elements of a list. User for loop.

  (\textbf{\color{BrickRed}File:} \kctt{print_prod.py})

\question Write a program that prints 'True' if a number is already present in a given
   list, 'False' otherwise. User for loop.

  (\textbf{\color{BrickRed}File:} \kctt{find.py})

\question Write a program that finds and prints the maximum element in a list. User for loop.

  (\textbf{\color{BrickRed}File:} \kctt{max_list.py})

\end{questions}
\end{document}